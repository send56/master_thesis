%====================================================================================================
\begin{abstractpage}{Abstract}

    %Simulation is a necessary tool of modern particle physics experiments, serving as a bridge between experiment and theory, therefore playing a critical role in both the analysis of experimental data and the validation of theoretical models. By accurately modeling particle interactions and detector responses, simulations enable reliable predictions of signal and background processes, optimization of detector performance, and evaluation of systematic uncertainties. thereby supporting the overall success of the research.


    The Large Hadron Collider (LHC), operated by CERN, is the highest-energy proton-proton collider in the world. One of its main detectors, ATLAS, is a general-purpose detector located at one of the collision points and is designed for studying fundamental physics searching for phenomena beyond the Standard Model. The Phase-II upgrade to the High-Luminosity LHC (HL-LHC), starting from autumn 2026, will require an upgrade of the ATLAS trigger system in response to the significantly increased luminosity. Along with this upgrade, the TGC (Thin Gap Chamber) detector, a part of the muon detector for forward-going muons, will have all its electronics renewed in HL-LHC. This study focuses on the development of a software simulator for the TGC Sector Logic, which reconstructs muon tracks from TGC hit information, and on its implementation and testing within the ATLAS software framework, Athena.

    The main challenge of the integration on Athena is that directly integrating the existing software algorithm would cause memory usage beyond the limit for distributed computing environment. In this research, an L0TGCSimulator was developed to provide the exact simulation of the Sector Logic behavior, and it has been successfully implemented within the Athena environment. This L0TGCSimulator is based on an existing bitwise simulator, which emulates the bit-level operation of the firmware logic on FPGA. After the implementation, an optimization for memory reduction was performed for the L0TGCSimulator, achieving about a 55\% reduction compared to the initial evaluated value. Finally, a Monte Carlo simulation with performance evaluation was carried out for the L0TGCSimulator. In addition, a simple emulator that emulates the muon trigger responses was developed to enable downstream developments in the muon trigger chain.

\end{abstractpage}
%====================================================================================================
