%====================================================================================================
\begin{abstractpage}{Abstract}
    Simulation is a necessary tool of modern particle physics experiments, serving as a bridge between experiment and theory, therefore playing a critical role in both the analysis of experimental data and the validation of theoretical models. By accurately modeling particle interactions and detector responses, simulations enable reliable predictions of signal and background processes, optimization of detector performance, and evaluation of systematic uncertainties. thereby supporting the overall success of the research.

    The Large Hadron Collider (LHC), operated by CERN, is the highest-energy proton-proton collider in the world. One of its main detectors, ATLAS, is a general-purpose detector located at one of the collision points and is designed for studying of fundamental physics searching for phenomena beyond the Standard Model. With the upcoming Phase-II upgrade to High-Luminosity LHC (HL-LHC) in 2030, the electronics in muon trigger system will be completely upgraded in response to the increased luminosity. As part of this upgrade, a software-based simulation of the Level-0 muon trigger firmware is required to support development and testing. 

    This study begins with a performance evaluation of the existing L0MuonEmulator, through which its limitations in precision for physics analysis were identified. To achieve a more comprehensive and higher-precision simulation, a new development of the L0MuonS1TGC simulation was processed based on an existing bitwise simulator, including its optimization and local implementation on ATLAS software framework. Finally, a Monte Carlo simulation and a performance evaluation were carried out.
\end{abstractpage}
%====================================================================================================
