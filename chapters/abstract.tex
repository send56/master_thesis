%====================================================================================================
\begin{abstractpage}{Abstract}

    %Simulation is a necessary tool of modern particle physics experiments, serving as a bridge between experiment and theory, therefore playing a critical role in both the analysis of experimental data and the validation of theoretical models. By accurately modeling particle interactions and detector responses, simulations enable reliable predictions of signal and background processes, optimization of detector performance, and evaluation of systematic uncertainties. thereby supporting the overall success of the research.


    The Large Hadron Collider (LHC), operated by CERN, is the highest-energy proton-proton collider in the world. One of its main detectors, ATLAS, is a general-purpose detector located at one of the collision points and is designed for studying of fundamental physics searching for phenomena beyond the Standard Model. With the upcoming Phase-II upgrade to High-Luminosity LHC (HL-LHC) in 2030, the ATLAS trigger system will be upgraded in response to the significantly increased luminosity. Along with this upgrade, the TGC (Thin Gap Chamber) detector, a part of the muon detector for forward-going muons, will have all its electronics renewed in HL-LHC. This study focuses on the development and implementation of a software simulator for the TGC Sector Logic, which is responsible for calculating the muon transverse momentum based on hits information on TGC, and on its integration and testing within the ATLAS software framework, Athena.

    The main problem of the integration on Athena is that directly integrating the existing software algorithm would cause a huge memory usage that substantially exceeds Athena’s limits. To address this problem, a simple emulator for the muon trigger was first developed to support downstream developments in the muon trigger chain. After that, in order to reproduce the trigger logic as the firmware does, and thereby to achieve the exact simulation of the logics in the Sector Logic, the implementation of the simulator for TGC Sector Logic was performed based on an existing bitwise simulator, which emulates the bit-level operation of the firmware logic on FPGA. The study includes an implementation on a local machine that replicates the Athena environment, along with an optimization for memory reduction. As a result of this optimization, memory consumption was reduced by 55\% compared to the initial evaluated value. Finally, a Monte Carlo simulation with performance evaluation was carried out for the implemented simulator based on the bitwise simulator.

\end{abstractpage}
%====================================================================================================
