%====================================================================================================
\chapter{Conclusion and Outlook} \label{ch:conclusion}
%====================================================================================================
The ongoing upgrade of the ATLAS detector for the High-Luminosity LHC operation will involve a comprehensive upgrade of the electronics for the endcap muon trigger system, including the TGC Sector Logic, which is responsible for muon track reconstruction and transverse momentum calculation. To support the simulation of the TGC Sector Logic, an appropriate simulator needs to be implemented within the ATLAS software framework, Athena.

This study developed two kind of simulation software for the Athena implementation. This first simulation software developed is a simple emulator, the L0MuonEmulator, in order to support downstream development in the trigger chain simulation. The emulator smears the $p_{\mathrm{T}}$ of muons and assumes a flat efficiency for muons above a $p_{\mathrm{T}}$ threshold across angular coverage of muon detector. Some ``holes'' in the detector acceptance caused by mechanical structures are excluded in the coverage, where the efficiency is set as zero. The efficiency of the L0MuonEmulator shows similar behaviour to the efficiency of the endcap muon trigger evaluated by the LHC Run~3 data. The emulator also gives similar plateau efficiency to the efficiency for the HL-LHC configuration of the barrel muon trigger estimated earlier. This L0MuonEmulator provides a useful framework to study the muon trigger spatial dependence and efficiency.

The second simulation software is a simulator running on the Athena environment, the L0TGCSimulator, which reproduces the firmware behaviour of the hardware L0 trigger for the endcap muon system. The firmware simulation is based on the existing simulator, the bitwise simulator, emulating the bit-level operation of the TGC Sector Logic firmware. In order to reduce the memory usage constrained by the offline computing environment, an optimization was performed through the simplification of the LUT storage structure, which reduced the memory usage from the expected 14~GB to approximately 6.3~GB, corresponding to a reduction of about 55\%. The efficiency of the implemented L0TGSimulator showed that the efficiency in wire segment was consistent with that of the stand-alone bitwise simulator. The strip segment, however, has an efficiency about 10\% lower than the that in the reference. Investigation suggests that this discrepancy originates from mismatches in input adaptation between the bitwise simulator and Athena. The exact reason is under investigation.

As an outlook of this study, the inefficiency in the strip segment is expected to be solved. For further memory reduction of the L0TGCSimulator, machine learning are being considered as potential replacements for the current reconstruction logic by look-up table. The machine learning, such as the artificial neural network, is widely used to generate various complicated patterns with small number of network nodes, which thereby consumes a small amount of memmory. 
