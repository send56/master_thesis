% Define shortcut commands

% Mathematical environments
\def \bi {\begin{itemize}}
\def \ei {\end{itemize}}
\def \be {\begin{linenomath*}\begin{equation}}
\def \ee {\end{equation}\end{linenomath*}}
\def \bs {\begin{linenomath*}\begin{equation} \left\{  \begin{aligned}}
\def \es {\end{aligned} \right. \end{equation}\end{linenomath*}}
\def \ba {\begin{equation} \begin{aligned}}
\def \ea {\end{aligned} \end{equation}}
\def \begm {\left(\begin{matrix}}
\def \endm {\end{matrix}\right)}

% Mathematical symbols
\def \dd  {{\mathrm d}}
\def \expe {\mathbb{E}}
\def \real {\mathbb{R}}
\def \e {\mathrm{e}}
\def \un  {\, \mathrm}
\def \ve {\mathbf}
\def \deg {\ensuremath{^{\circ}}}
\newcommand{\mypow}[2]{\ensuremath{{#1}\times 10^{#2}}\xspace}

% Physics
\def \bsll {\ensuremath{b\to s\ell^+\ell^-}\xspace}
\def \BKll {\ensuremath{B\to K^{(*)}\ell^+\ell^-}\xspace}
\def \BKpll {\ensuremath{B^+\to K^{+}\ell^+\ell^-}\xspace}
\def \BKpsll {\ensuremath{B^+\to K^{*+}\ell^+\ell^-}\xspace}
\def \BKpsll {\ensuremath{B^0\to K^{0}\ell^+\ell^-}\xspace}
\def \BKzsll {\ensuremath{B^0\to K^{*0}\ell^+\ell^-}\xspace}
\def \BKnn {\ensuremath{B\to K\nu\bar{\nu}}\xspace}
\def \BKpnn {\ensuremath{B^+\to K^+\nu\bar{\nu}}\xspace}
\def \BKznn {\ensuremath{B^0\to K^0_{\mathrm S}\nu\bar{\nu}}\xspace}
\def \BKzznn {\ensuremath{B^0\to K^0\nu\bar{\nu}}\xspace}
\def \BKjpsi {\ensuremath{B\to K\jpsi}\xspace}
\def \BKjpsimumu {\ensuremath{B\to K\jpsi(\to\mumu)}\xspace}
\def \BKpjpsi {\ensuremath{B^+\to K^+\jpsi}\xspace}
\def \BKzjpsi {\ensuremath{B^0\to K^0_{\mathrm S}\jpsi}\xspace}
\def \bsnn {\ensuremath{b\to s\nu\bar{\nu}}\xspace}
\def \mbc {\ensuremath{M_\mathrm{bc}}\xspace}
\def \nsig {\ensuremath{n_\mathrm{sig}}\xspace}
\def \nbkg {\ensuremath{n_\mathrm{bkg}}\xspace}
\def \Br {\ensuremath{\mathrm{Br}}\xspace}
\def \babar {Babar\xspace}
\def \qrec {\ensuremath{q_{\mathrm{rec}}^2}\xspace}
\def \qq {\ensuremath{q^2}\xspace}
\def \gevcccc {\ensuremath{\,\mathrm{GeV^2/c^4}}\xspace}
\def \effsig {\ensuremath{\varepsilon_{\mathrm{sig}}}\xspace}
\def \pt {\ensuremath{p_{\mathrm{T}}}\xspace}
\def \dr {\ensuremath{dr}\xspace}
\def \dz {\ensuremath{dz}\xspace}
\def \bdto {\ensuremath{\mathrm{BDT_1}}\xspace}
\def \bdtt {\ensuremath{\mathrm{BDT_2}}\xspace}
\def \bdtc {\ensuremath{\mathrm{BDT_c}}\xspace}
\def \CLs {\ensuremath{\mathrm{CL_s}}\xspace}
\def \esig {\ensuremath{\tilde{\varepsilon}_{\mathrm{sig}}}\xspace}

% Numbers
\def \nlumion {189} % 189.2
\def \nlumioff {18} % 18.0
\def \nlumitot {208}
\def \nlumimc {1000}
\def \nlumionpartial {63}
\def \nlumioffpartial {9}
\def \lumion {\nlumion\invfb}
\def \lumioff {\nlumioff\invfb}
\def \lumionpartial {\nlumionpartial\invfb}
\def \lumioffpartial {\nlumioffpartial\invfb}
\def \lumitot {\nlumitot\invfb}
\def \lumimc {\nlumimc\invfb}
\def \lumimctest {800\invfb}
\def \limitKp {\mypow{1.0}{-5}}
\def \limitKz {\mypow{1.8}{-5}}
\def \limitKppartial {\mypow{4.1}{-5}}
\def \nsignalmc {\mypow{8}{6}}
\def \nsignalmctrain {\mypow{4}{6}}
\def \nsignalmctest {\mypow{4}{6}}

% Misc
\newcommand{\autocite}[1]{\cite{#1}}
\newcommand{\myref}[1]{Ref.~\cite{#1}}

\newcommand{\Tstrut}{\rule{0pt}{2.6ex}}         % = `top' strut
\newcommand{\Bstrut}{\rule[-0.9ex]{0pt}{0pt}}   % = `bottom' strut
\newcommand{\toprule}{\hline\Bstrut}
\newcommand{\midrule}{\hline\Tstrut\Bstrut}
\newcommand{\bottomrule}{\hline\Tstrut}

\newcommand{\textfrac}[2]{%
	$\frac{\text{#1}}{\text{#2}}$
}

% Table
\newcommand{\tab}[4]{%
    \begin{table}[bt]
    \caption{#4}
    \begin{center}
    \begin{tabular}{#2}
    #3
    \end{tabular}
    \end{center}
    \label{tab:#1}
    \end{table}
}

% Table
\newcommand{\tabs}[6]{%
    \begin{table}[t]
    \caption{#6}
    \begin{center}
    \begin{tabular}[t]{#2}
    #3
    \end{tabular}
    \begin{tabular}[t]{#4}
    #5
    \end{tabular}
    \end{center}
    \label{tab:#1}
    \end{table}
}

% Deal with missing figures which are not directly included in the repository
% Taken from https://tex.stackexchange.com/questions/39982/use-default-figure-if-file-is-missing
\newcommand{\noimage}{%
  \setlength{\fboxsep}{-\fboxrule}%
  \fbox{\phantom{\rule{10pt}{10pt}}Figure missing\phantom{\rule{10pt}{10pt}}}% Framed box
}
\let\includegraphicsoriginal\includegraphics
\renewcommand{\includegraphics}[2][width=\textwidth]{\IfFileExists{#2}{\includegraphicsoriginal[#1]{#2}}{\noimage}}


% Simple figure
\newcommand{\fig}[4]{%
    \begin{figure}[tb]
    \centering
    \includegraphics[width=#2\textwidth]{#3}
    \caption{#4}
    \label{fig:#1}
    \end{figure}
}

% Figure with two subfigures
\newcommand{\figs}[6]{%
    \begin{figure}[tb]
    \centering
    \includegraphics[width=#2\textwidth]{#3}
    \includegraphics[width=#4\textwidth]{#5}
    \caption{#6}
    \label{fig:#1}
    \end{figure}
}

% Figure with arbitrary number of subfigures
\newcommand{\figss}[3]{%
    \begin{figure}[tb]
    \centering
    {#2}
    \caption{#3}
    \label{fig:#1}
    \end{figure}
}

